\documentclass[11pt,a4paper]{article}
\usepackage{../ma426}
\semester{Fall}
\year{2019}
\subtitlenumber{1}
\author{刘逸灏 (515370910207)}

\begin{document}
\maketitle

\section{2.1/4}
\begin{problem}
  求曲线
  $$
    \left\{
    \begin{aligned}
      x^2+y^2+z^2=1 & ,\quad z\geqslant0, \\
      x^2+y^2=x     &
    \end{aligned}
    \right.
  $$
  的参数方程
\end{problem}

$$x^2-x+\frac{1}{4}+y^2=\frac{1}{4},$$
$$\left(x-\frac{1}{2}\right)^2+y^2=\frac{1}{4}.$$

设$x=\dfrac{1}{2}+\dfrac{1}{2}\cos t$, $y=\dfrac{1}{2}\sin t$, 并代入第一个方程
$$\left(\frac{1}{2}+\frac{1}{2}\cos t\right)^2+\left(\frac{1}{2}\sin t\right)^2+z^2=1,\quad z\geqslant0,$$
$$z^2=1-\frac{1}{4}-\frac{1}{2}\cos t-\frac{1}{4}(\cos^2t+\sin^2t)=\frac{1}{2}-\frac{1}{2}\cos t=\sin^2\frac{t}{2},\quad z=\sin\frac{t}{2}.$$

故曲线的参数方程为
$$\mathbf{r}(t)=\left(\frac{1}{2}+\frac{1}{2}\cos t,\frac{1}{2}\sin t,\sin\frac{t}{2}\right).$$

\section{2.1/6}
\begin{problem}
  设空间$E^3$中一条正则参数曲线$\mathbf{r}(t)$的切向量$\mathbf{r}'(t)$与一个固定的方向向量$\boldsymbol\alpha$垂直. 证明: 该曲线落在一个平面内.
\end{problem}

假设该曲线不落在一个法向量为$\boldsymbol\alpha$的平面上, 即存在点$t_0$使得$\mathbf{r}'(t_0)$不与该平面平行, 显然这与题设条件矛盾, 故该曲线落在该平面内.

\section{2.2/1}
\begin{problem}
  求下列曲线在指定范围内的弧长:
  \begin{enumerate}
    \setcounter{enumi}{3}
    \item 掖物线$\mathbf{r}(t)=(\cos t, \log(\sec t+\tan t)-\sin t),\quad [0,t_0];$
    \item 曲线$y=x^2/2a,z=x^3/6a^2$在原点$O(0,0,0)$和点$p(x_0,y_0,z_0)$之间.
  \end{enumerate}
\end{problem}

\subsection*{(4)}

$$\mathbf{r}(t)=(\cos t, \log(\sec t+\tan t)-\sin t),$$
$$\mathbf{r}'(t)=\left(-\sin t, \frac{\tan t\sec t+\sec^2t}{\sec t+\tan t} -\cos t\right)=(-\sin t,\sec t-\cos t),$$
$$|\mathbf{r}'(t)|=\sqrt{\sin^2t+(\sec t-\cos t)^2}=\sqrt{\sin^2t+\sec^2t-2+\cos^2t}=|\tan t|,$$
$$\int_0^{t_0}|\mathbf{r}'(t)|dt=\int_0^{t_0}|\tan t|dt=-\ln\cos t_0,\quad t_0<\frac{\pi}{2}.$$

\subsection*{(5)}

$$\mathbf{r}(t)=\left(t,\frac{t^2}{2a},\frac{t^3}{6a^2}\right),
  \quad t\in[0,x_0].$$
$$\mathbf{r}'(t)=\left(1,\frac{t}{a},\frac{t^2}{2a^2}\right),$$
$$|\mathbf{r}'(t)|=\sqrt{1+\frac{t^2}{a^2}+\frac{t^4}{4a^4}}=\frac{t^2}{2a^2}+1,$$
$$\int_0^{x_0}|\mathbf{r}'(t)|dt=\int_0^{x_0}\left(\frac{t^2}{2a^2}+1\right)dt=\frac{x_0^3}{6a^2}+x_0.$$

\section{2.2/5}
\begin{problem}
  求曲线$\mathbf{r}(t)=(t^3/3,t^2/2,t)$上其切线平行于平面$x+3y+2z=0$的点.
\end{problem}

$$\mathbf{r}(t)=(t^3/3,t^2/2,t),$$
$$\mathbf{r}'(t)=(t^2,t,1).$$

在平面$x+3y+2z=0$上取两不平行向量$(0,2,-3)$, $(2,0,-1)$可得到平面的一个法向量
$$\mathbf{n}=(0,2,-3)\times(2,0,-1)=(-2,-6,-4)\propto(1,3,2).$$

当$\mathbf{r}'(t)$与平面平行时, 与平面法向量垂直
$$\mathbf{r}'(t)\cdot\mathbf{n}=t^2+3t+2=0,$$
$$t=-1\quad\text{或}\quad t=-2.$$

故切线平行于平面的点为$\left(-\dfrac{1}{3},\dfrac{1}{2},-1\right)$和$\left(-\dfrac{8}{3},2,-2\right)$.


\end{document}
