\documentclass[11pt,a4paper]{article}
\usepackage{../ma426}
\semester{Fall}
\year{2019}
\subtitlenumber{8}
\author{刘逸灏 (515370910207)}

\begin{document}

\maketitle

\section{4.2/6}
\begin{problem}
求下列曲面上的渐进曲线:
\begin{enumerate}
  \addtocounter{enumi}{1}
  \item 双曲抛物面: $\mathbf{r}=\left(\dfrac{u+v}{2},\dfrac{u-v}{2},uv\right)$.
\end{enumerate}
\end{problem}

\subsection*{(2)}
$$\mathbf{r}_u(u,v)=\left(\frac{1}{2},\frac{1}{2},v\right),$$
$$\mathbf{r}_v(u,v)=\left(\frac{1}{2},-\frac{1}{2},u\right),$$
$$\mathbf{r}_u\times\mathbf{r}_v=\left(\frac{u+v}{2},-\frac{u-v}{2},-\frac{1}{2}\right),$$
$$|\mathbf{r}_u\times\mathbf{r}_v|=\sqrt{\frac{1}{4}(u+v)^2+\frac{1}{4}(u-v)^2+\frac{1}{4}}=\frac{\sqrt{2u^2+2v^2+1}}{2},$$
$$\mathbf{n}(u,v)=\frac{\mathbf{r}_u\times\mathbf{r}_v}{|\mathbf{r}_u\times\mathbf{r}_v|}=\frac{1}{\sqrt{2u^2+2v^2+1}}(u+v,v-u,-1),$$
$$\mathbf{r}_{uu}(u,v)=(0,0,0),$$
$$\mathbf{r}_{uv}(u,v)=(0,0,1),$$
$$\mathbf{r}_{vv}(u,v)=(0,0,0),$$
$$L(u,v)=\mathbf{r}_{uu}\cdot\mathbf{n}=0,$$
$$M(u,v)=\mathbf{r}_{uv}\cdot\mathbf{n}=-\frac{1}{\sqrt{2u^2+2v^2+1}},$$
$$N(u,v)=\mathbf{r}_{vv}\cdot\mathbf{n}=0.$$
第二基本形式为
$$II=L(u,v)(du)^2+2M(u,v)dudv+N(u,v)(dv)^2=-\frac{1}{\sqrt{2u^2+2v^2+1}}dudv.$$
$II=0$的解为$du=0$或$dv=0,$
故渐进曲线为$u=u_0$和$v=v_0$.

\section*{定理 4.3.2}
\begin{problem}
Weingarten 映射$W$是从切空间$T_pS$到它自身的自共轭映射, 即对于曲面$S$在点$u,v$的任意两个切方向$d\mathbf{r}$和$\delta\mathbf{r}$, 下面的公式
$$W(d\mathbf{r})\cdot\delta\mathbf{r}=d\mathbf{r}\cdot W(\delta\mathbf{r})$$
成立.
\end{problem}
设
$$d\mathbf{r}=\mathbf{r}_udu+\mathbf{r}vdv,\quad \delta\mathbf{r}=\mathbf{r}_u\delta u+\mathbf{r}_v\delta v.$$
则
$$W(d\mathbf{r})=-(\mathbf{n}_udu+\mathbf{n}_vdv),\quad W(\delta\mathbf{r})=-(\mathbf{n}_u\delta u+\mathbf{n}_v\delta v).$$
\begin{align*}
  W(d\mathbf{r})\cdot\delta\mathbf{r}
   & =-(\mathbf{n}_udu+\mathbf{n}_vdv)\cdot(\mathbf{r}_u\delta u+\mathbf{r}_v\delta v) \\
   & =Ldu\delta u+M(du\delta v+dv\delta u)+Ndv\delta v                                 \\
   & =(\mathbf{r}_udu+\mathbf{r}vdv)\cdot-(\mathbf{n}_u\delta u+\mathbf{n}_v\delta v)  \\
   & =d\mathbf{r}\cdot W(\delta\mathbf{r}).
\end{align*}

\section{4.3/4}
\begin{problem}
证明: 曲面$S$上任意一点$p$的某个邻域内都有正交参数系$(u,v)$, 使得参数曲线在点$p$处的切方向是曲面$S$在该点的两个彼此正交的主方向.
\end{problem}
设曲面$S$的正交参数系为$(u,v)$, 则在点$p(u_0,v_0)$处的两个正交的主方向单位向量为
$$\mathbf{e}_1=\xi_1\mathbf{r}_u(u_0,v_0)+\eta_1\mathbf{r}_v(u_0,v_0),$$
$$\mathbf{e}_2=\xi_2\mathbf{r}_u(u_0,v_0)+\eta_2\mathbf{r}_v(u_0,v_0).$$
现只需找到变换$T:(u,v)\longrightarrow(u',v')$使得$\mathbf{r}_{u'}(u_0,v_0)=\mathbf{e}_1$, $\mathbf{r}_{v'}(u_0,v_0)=\mathbf{e}_2$. 令
$$T:\quad u=\xi_1u'+\xi_2v',\quad v=\eta_1u'+\eta_2v',$$
$$\frac{\partial(u,v)}{\partial(u',v')}=
  \begin{vmatrix}
    \dfrac{\partial u}{\partial u'} & \dfrac{\partial v}{\partial u'} \\
    \dfrac{\partial u}{\partial v'} & \dfrac{\partial v}{\partial v'}
  \end{vmatrix}=\begin{vmatrix}
    \xi_1 & \eta_1 \\ \xi_2 & \eta_2
  \end{vmatrix}\neq0,$$
$$\mathbf{r}_{u'}=\frac{\partial u}{\partial u'}\mathbf{r}_u(u_0,v_0)+\frac{\partial v}{\partial u'}\mathbf{r}_v(u_0,v_0)=\xi_1\mathbf{r}_u(u_0,v_0)+\eta_1\mathbf{r}_v(u_0,v_0)=\mathbf{e}_1,$$
$$\mathbf{r}_{v'}=\frac{\partial u}{\partial v'}\mathbf{r}_u(u_0,v_0)+\frac{\partial v}{\partial v'}\mathbf{r}_v(u_0,v_0)=\xi_2\mathbf{r}_u(u_0,v_0)+\eta_2\mathbf{r}_v(u_0,v_0)=\mathbf{e}_2.$$
故变换成立, $(u',v')$即为需证的正交参数系.

\end{document}
